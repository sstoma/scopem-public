% Created 2016-04-18 Mon 18:36
\documentclass[11pt]{article}
\usepackage[utf8]{inputenc}
\usepackage[T1]{fontenc}
\usepackage{fixltx2e}
\usepackage{graphicx}
\usepackage{longtable}
\usepackage{float}
\usepackage{wrapfig}
\usepackage{rotating}
\usepackage[normalem]{ulem}
\usepackage{amsmath}
\usepackage{textcomp}
\usepackage{marvosym}
\usepackage{wasysym}
\usepackage{amssymb}
\usepackage{hyperref}
\tolerance=1000
\author{Simon Noerrelykke \& Szymon Stoma}
\date{\textit{<2015-12-08 Tue>}}
\title{Introduction to FIJI - workflows v. 1.2}
\hypersetup{
  pdfkeywords={fiji, imagej, teaching},
  pdfsubject={},
  pdfcreator={Emacs 25.0.50.4 (Org mode 8.2.10)}}
\begin{document}

\maketitle
\tableofcontents

\section{working with single channel images}
\label{sec-1}
\subsection{opening the image and basic info}
\label{sec-1-1}
\begin{enumerate}
\item file > open > HeLa (IJ:26.2)
\item image > show info (ctrl-i) (IJ:28.3)
\item analyze > tools > scale bar (IJ:30.14.6)
\end{enumerate}

\subsection{opening the image and basic drawing}
\label{sec-1-2}
\begin{enumerate}
\item \textbf{target:} "convert" [geometry-source.tif] into [geometry-target.tif]
\item file > open : [geometry-source.tif] (IJ:26.2)
\item use following tools to convert source into target:
\begin{enumerate}
\item\relax [palete > straight line] (IJ:19.12.1)
\begin{itemize}
\item try shift
\end{itemize}
\item\relax [palete > color picker] (IJ:19.11)
\begin{itemize}
\item e.g. choose black as foreground
\item try Draw (d) (IJ:27.9-10)
\end{itemize}
\item\relax [palete > flood fill tool] (IJ:19.16)
\begin{itemize}
\item use again: color or picker to choose right color for filling
\end{itemize}
\item\relax [palete > wand tracking tool] (IJ:19.7)
\begin{itemize}
\item use to select shapes
\item apply Copy (ctrl-c), Paste (ctrl-v), Cut (ctrl-x)
\item alternatively select, move selection, choose color and try Fill (f), Draw (d), Clear tools (IJ:27.2 IJ:27.9-10)
\end{itemize}
\item\relax [palete > rectangular selection] (IJ:19.1)
\begin{enumerate}
\item select a rectangular shape around blue rectangle
\item apply Copy (ctrl-c), Paste (ctrl-v)
\end{enumerate}
\end{enumerate}
\end{enumerate}

\subsection{copy pasting from image to image}
\label{sec-1-3}
\begin{enumerate}
\item \textbf{target:} create a figure illustrating the difference in C. elegans
phenotype (see elegance-fig.tif) based on elegans.tif example
\item file > new > image (IJ:26.1)
\begin{itemize}
\item Type: 8bit
\item Slices: 1
\item Fill with: black
\end{itemize}
\item file > save as > tif : my-elegance-fig.tif (IJ:26.10.1)
\item\relax [palete > color picker] (IJ:19.11)
\begin{itemize}
\item choose white as foreground and black as background
\end{itemize}
\item file > open : [elegance.tif] (IJ:26.2)
\item find two different phenotypes of a worm; for each worm
\begin{enumerate}
\item select worm with [palete > rectangular selection]
(IJ:19.1)
\item edit > selection > add to manager (ctrl-t) (IJ:27.12.22)
\begin{itemize}
\item analyze > tools > ROI manager
\end{itemize}
\end{enumerate}
\item prepare the template as in elegance example (see elegance-fig.tif)
\begin{enumerate}
\item\relax [palete > straight line] (ctrl-click to select \textbf{straight} mode) (IJ:19.2)
\begin{itemize}
\item use line selection to create two horizontal lines (use shift
while drawing)
\item create line; Draw (d); move line down; Draw (d); add to
selection for future
\end{itemize}
\item\relax [palete > text tool] (IJ:19.8)
\begin{itemize}
\item add text for "GFP" and "Brightfield"
\item confirm with ctrl-d (or edit > draw)
\end{itemize}
\item use ROI manager to put ROIs into right places (analogous to
elegance-fig.tif); for each ROI from original image:
\begin{itemize}
\item paste ROI into new image
\item move it to desired location
\item add it to the ROI manager
\end{itemize}
\end{enumerate}
\item copy paste selected worms to the new images (two channel per worm)
\begin{itemize}
\item use ROI to find a worm position
\item choose channel: GFP or Brightfield
\item copy selection
\item choose my-elegance-fig.tif window
\item use ROI to select desired paste location
\item paste
\item repeat procedure for all four images
\end{itemize}
\item file > save (IJ:26.10.1)
\end{enumerate}

\subsection{file handling and non-invasive editing}
\label{sec-1-4}
\begin{enumerate}
\item file > open samples > blobs (shift-b) (IJ:26.4)
\item analyze > tools > scale bar (IJ:30.14.6)
\begin{itemize}
\item set color, background, position
\item check: overlay, bold
\end{itemize}
\item file > save as > tiff : blobs1.tif (IJ:26.10.1)
\begin{itemize}
\item saving the image as tiff keeps the scale bar as an overlay, so pixel-values below it are kept
\end{itemize}
\end{enumerate}

\subsection{file handling and invasive editing}
\label{sec-1-5}
\begin{enumerate}
\item file > open : [blobs1.tif] (IJ:26.2)
\item image > info (i) (IJ:28.3)
\item image > overlay > remove overlay (IJ:28.14.7)
\item analyze > set scale  (IJ:30.8)
\begin{itemize}
\item set scale such as image is 100 um long in each dimension
\end{itemize}
\item analyze > tools > scale bar (IJ:30.14.6)
\begin{itemize}
\item set color, background, position
\item check: NO overlay
\end{itemize}
\item file > save as > tiff : blobs2.tif (IJ:26.10.1)
\begin{itemize}
\item load again to check the difference (e.g. check info)
\end{itemize}
\end{enumerate}

\subsection{8-bit and 16-bit pseudocolor images}
\label{sec-1-6}
\begin{enumerate}
\item file > open samples > blobs (shift-b) (IJ:26.4)
\item inspect pixel values (IJ:p.28:Toolbar)
\item image > lookup table > invert LUT (IJ:28.15.1)
\item analyze > tools > calibration bar (IJ:30.14.7)

\item file > open > [blobs16bit.tif] (IJ:26.2)
\item inspect pixel values (IJ:p.28:Toolbar)
\begin{itemize}
\item alternatively use Pixel Inspector (IJ:19.20)
\end{itemize}
\item analyze > tools > calibration bar (IJ:30.14.7)

\item file > open samples > blobs (shift-b) (IJ:26.4)
\item add arrow in overlay (IJ:19.13)
\begin{itemize}
\item play with the look of the arrow (i.e. colors, thickness)
\item indicate a blob which you like
\item confirm by ctrl-b (IJ:19.8)
\end{itemize}
\item add arrow in draw (IJ:19.13)
\begin{itemize}
\item confirm by ctrl-d (IJ:19.8)
\end{itemize}
\item mark a part of image with overlay brush (IJ:19.18)
\begin{itemize}
\item play with transparency
\item cover blobs marked by arrows with red paint
\end{itemize}
\end{enumerate}

\subsection{pseudocolor image to RGB conversion}
\label{sec-1-7}
\begin{enumerate}
\item file > open samples > blobs (shift-b) (IJ:26.4)
\item image > duplicate (IJ:28.9)
\item rename images: left and right
\begin{itemize}
\item image > rename (IJ:28.10)
\end{itemize}
\item change color lookup table for left and right
\begin{itemize}
\item Image > Look up table > Fire | green
\end{itemize}
\item duplicate both images (IJ:28.9)
\begin{itemize}
\item name them left-1 and right-1
\end{itemize}
\item image > stack > tools > combine (IJ:28.6.15.1)
\begin{itemize}
\item select left and right
\end{itemize}
\item select left-1
\begin{itemize}
\item image > type > Color RGB (IJ:7)
\item check with picker what are the values in pixels (IJ:p.28:Toolbar)
\end{itemize}
\item select left-2
\begin{itemize}
\item image > type > Color RGB (IJ:7)
\end{itemize}
\item image > stack > tools > combine (IJ:28.6.15.1)
\begin{itemize}
\item select left-1 and right-1
\end{itemize}
\end{enumerate}

\subsection{color spaces}
\label{sec-1-8}
\begin{enumerate}
\item file > open : [blob-combined.tif] (IJ:26.2)
\item plugins > Color Inspector 3D
\item switch "Display mode" to histogram
\item play with color: rotation | saturation | brightness
\begin{itemize}
\item compare RGB space with e.g. LAB
\end{itemize}
\end{enumerate}

\subsection{histogram operations}
\label{sec-1-9}
\begin{enumerate}
\item file > open samples > blobs (shift-b) (IJ:26.4)
\item image > lookup tables > invert LUT (IJ:28.15.1)
\item analyze > histogram (IJ:30.10)
\begin{itemize}
\item mark live
\end{itemize}
\item select line tool (IJ:19.2) (IJ:p.28:Interface)
\item analyze > plot profile (IJ:30.11)
\begin{itemize}
\item mark Live
\item move line
\item double click on line icon
\item change line thickness (move line)
\end{itemize}
\item edit > selection > select none
\item image > adjust > brightness/contrast (IJ:28.2.1)
\begin{itemize}
\item play with settings to achieve white blobs and black background (actually you almost thresholded image)
\item apply when finished (note: pixel values are altered)
\end{itemize}

\item file > open samples > M51 galaxy (IJ:26.4)
\item (2x) image > duplicate (IJ:28.9)
\item (3x) analyze > histogram (IJ:30.10)
\begin{itemize}
\item mark live
\item check log button
\end{itemize}
\item on each image
\begin{itemize}
\item process > enhance contrast (IJ:29.5)
\begin{itemize}
\item equalize checked
\end{itemize}
\item process > enhance contrast (IJ:29.5)
\begin{itemize}
\item normalize checked
\end{itemize}
\item process > enhance local contrast (CLAHE)
\begin{itemize}
\item \url{http://fiji.sc/wiki/index.php/Enhance_Local_Contrast_(CLAHE)}
\end{itemize}
\end{itemize}
\item compare the results
\end{enumerate}

\subsection{using 16 bit images to increase precision}
\label{sec-1-10}
\begin{enumerate}
\item file > open samples > M51 galaxy (IJ:26.4)
\item image > lookup tables > fire (IJ:28.15)
\item image > duplicate (IJ:28.9)
\item (2x) analyze > histogram (IJ:30.10)
\begin{itemize}
\item mark live
\item check log button
\end{itemize}
\item on first image:
\begin{itemize}
\item image > type > 8 bit
\item process > enhance contrast (IJ:29.5)
\begin{itemize}
\item equalize checked
\end{itemize}
\end{itemize}
\item on second image:
\begin{itemize}
\item process > enhance contrast (IJ:29.5)
\begin{itemize}
\item equalize checked
\end{itemize}
\end{itemize}
\item for both img.: image > lookup tables > fire (IJ:28.15)
\item select center of the galaxy with [palete > rectangular
selection] (IJ:19.1)
\item edit > selection > add to manager (ctrl-t) (IJ:27.12.22)
\begin{itemize}
\item analyze > tools > ROI manager
\item switch window to second image
\item click ROI you just added that it appears in the second image
\end{itemize}
\item analyze > set measurements (IJ:30.7)
\begin{itemize}
\item check: mean gray value / standard deviation
\end{itemize}
\item for both img.: analyze > measure (m) (IJ:29.12.1)
\item for both img.: image > lookup tables > glasbey (IJ:28.15)
\item compare the results
\end{enumerate}

\section{working with multichannel images}
\label{sec-2}
\subsection{create new image}
\label{sec-2-1}
\begin{enumerate}
\item \textbf{target:} create a sketch of a cell as in [cell.tif]
\item file > new > image (IJ:26.1)
\begin{itemize}
\item Type: 8bit
\item Slices: 2
\item Fill with: black
\end{itemize}
\item\relax [palete > brush] (IJ:19.4)
\begin{itemize}
\item right click on the [palete > brush]
\begin{itemize}
\item uncheck: "Paint in overlay"
\end{itemize}
\end{itemize}
\item\relax [palete > color picker] (IJ:19.11)
\begin{itemize}
\item choose red as foreground
\end{itemize}
\item Draw an outline of a cell ;)
\item image > color > channel tool (ctrl-z) (IJ:28.5.3)
\begin{itemize}
\item make composite (IJ:28.5.5)
\end{itemize}
\item Switch a slice with a slider
\item\relax [palete > color picker] (IJ:19.11)
\begin{itemize}
\item choose blue as foreground
\end{itemize}
\item Draw cell nuclei
\item\relax [palete > brush] (IJ:19.4)
\begin{itemize}
\item right click on the [palete > brush]
\begin{itemize}
\item check: "Paint in overlay"
\end{itemize}
\end{itemize}
\item\relax [palete > color picker] (IJ:19.11)
\begin{itemize}
\item choose blue as foreground
\end{itemize}
\item Draw cell vesicles
\item Inspect pixel values (IJ:p.28:Toolbar)
\begin{itemize}
\item alternatively use Pixel Inspector (IJ:19.20)
\item switch a slice with a slider
\item move inspector between: outline | nuclei | vesicles
\end{itemize}
\item image > color > channel tool (ctrl-z) (IJ:28.5.3)
\begin{itemize}
\item switch between composite | color | grey
\end{itemize}
\item file > save as > tif : my-cell.tif (IJ:26.10.1)
\end{enumerate}

\subsection{composite images - splitting and merging}
\label{sec-2-2}
\begin{enumerate}
\item file > open samples > fluorescent cells (IJ:26.4)
\item image > color > arrange channels 
\begin{itemize}
\item click on New 1, and select magenta
\end{itemize}
\item image > color > split channels (IJ:28.5.1)
\item merge channels to composite
\begin{itemize}
\item image > color > channels tool (shift-z) (IJ:28.7.5)
\item check "create composite" box (IJ:28.5.2)
\end{itemize}
\item color blindness
\begin{itemize}
\item image > color > simulate color blindness
\item image > color > dichromacy
\end{itemize}
\end{enumerate}

\subsection{composite images - individual channel corrections}
\label{sec-2-3}
\begin{enumerate}
\item \textbf{target:} create an image with brightfield / red / cyan channels
which shows locations with strongest expression of these fluorophores
\item file > open samples > neuron (IJ:26.4)
\item image > color > channel tool (IJ:28.7.5)
\item for each channel 
\begin{itemize}
\item adjust contrast such as final image "conveys the message" (IJ:28.2.1)
\end{itemize}
\end{enumerate}

\subsection{microscopy stacks handling}
\label{sec-2-4}
\begin{enumerate}
\item \textbf{target:} make a video and a picture for publication showing progression of mitosis in time (see: mitosis-montage.tif)
\item open : [mitosis-mixedStack.tif] (IJ:26.2)
\item image > hyperstacks > reorder hyperstacks
\begin{itemize}
\item swap z with t
\end{itemize}
\item image > stacks > tools > make substack (IJ:28.6.15.7)
\item image > duplicate (IJ:28.9)
\begin{itemize}
\item check duplicate hyperstack
\end{itemize}
\item image > color > channels tool (shift-z) (IJ:28.7.5)
\begin{itemize}
\item split channels (IJ:28.5.1)
\end{itemize}
\item for both C1 and C2 images
\begin{itemize}
\item choose LUT Fire (IJ:19.7)
\item create channel label (IJ:19.8)
\item image > type > color RGB (IJ:7)
\end{itemize}
\item for original two channel image
\begin{itemize}
\item image > type > color RGB (IJ:7)
\end{itemize}
\item image > stacks > tools > combine (IJ:28.6.15.1)
\begin{itemize}
\item choose left: 2 channel image
\item choose right: C1
\end{itemize}
\item image > stacks > tools > combine (IJ:28.6.15.1)
\begin{itemize}
\item choose left: merged image
\item choose right: C2
\end{itemize}
\item image > stacks > series labeller
\begin{itemize}
\item select time and other options; use preview to peek
\end{itemize}
\item file > save as > avi (IJ:26.10.1)
\begin{itemize}
\item e.g. 10 frames per sec.
\end{itemize}
\item image > stacks > make montage (IJ:28.6.8)
\begin{itemize}
\item play with settings:
\begin{itemize}
\item columns 1
\item rows 5
\item increment 12
\end{itemize}
\end{itemize}
\item file > save as > tiff (IJ:26.10.1)
\end{enumerate}

\section{image processing: thresholding \& filters}
\label{sec-3}
\subsection{basic concept of thresholding}
\label{sec-3-1}
\begin{enumerate}
\item \textbf{target:} Threshold blob.gif image to create a mask enabling
segmentation of blob-like structures (i.e. create an image where the
values of all pixels not belonging to blob-like structures are
set to 0, and all other pixel values are equal to 255).
\item file > open samples > blobs (IJ:26.4)
\item image > lookup table > invert LUT (IJ:28.15.1)
\item image > duplicate (IJ:28.9)
\item image > adjust > threshold (IJ:28.2.4)
\begin{itemize}
\item set up sliders and Dark background checkbox and threshold the
image:
\begin{itemize}
\item use 126 value as threshold
\end{itemize}
\item note: pixel values are altered
\end{itemize}
\item (optional) image > adjust > auto threshold
\end{enumerate}

\subsection{basic concept of filtering: binary filters}
\label{sec-3-2}
\begin{enumerate}
\item open : [blobs-thr.tif] (IJ:26.2)
\item image > duplicate (IJ:28.9)
\item process > binary > watershed (IJ:29.8.12)
\item process > noise > remove outliers (IJ:29.6.5)
\begin{itemize}
\item check preview
\item list bright
\end{itemize}
\item (process > find maxima) (IJ:29.4)
\item (2x) process > binary > erode (IJ:29.8.3)
\item process > find edges (IJ:29.3)
\item process > binary > fill holes (IJ:29.8.8)
\end{enumerate}

\subsection{basic concept of filters: sharpen}
\label{sec-3-3}
\begin{enumerate}
\item file > open samples > hela cells (IJ:26.4)
\item process > filters > unsharp mask (IJ:29.11.8)
\begin{itemize}
\item check how it behaves when image is composite / color (use channel tool for this purpose) (IJ:28.7.5)
\end{itemize}
\end{enumerate}

\subsection{basics mathematical operations on single image}
\label{sec-3-4}
\begin{enumerate}
\item open : [f2.tif] (IJ:26.2)
\item process > math > add (IJ:29.9.1)
\begin{itemize}
\item play with different functions
\item check what happens when image is 32 bit
\end{itemize}
\end{enumerate}
\subsection{beyond the limits of bits}
\label{sec-3-5}
\begin{enumerate}
\item \textbf{target:} Perform consecutive addition and subtraction of the
same value from an image. Compare the results with original image.
\item open : [spooked$_{\text{16bit}}$.tif] (IJ:29.9.1)
\item image > duplicate (IJ:28.9)
\begin{itemize}
\item work on the copy
\end{itemize}
\item process > math > add (IJ:29.9.1)
\begin{itemize}
\item add value: 600
\end{itemize}
\item process > math > subtract (IJ:29.9.2)
\begin{itemize}
\item subtract value: 600
\end{itemize}
\item comapare the original and the copy: are they the same? What has happened?
\end{enumerate}
\subsection{basics mathematical operations on two images}
\label{sec-3-6}
\begin{enumerate}
\item open : [f2.tif] and [f1.tif] (IJ:26.2)
\item process > calculator (IJ:29.13)
\begin{itemize}
\item start with add
\item check what happens when image is 32 bit
\item play with different functions
\end{itemize}
\end{enumerate}
\subsection{seeing JPEG artifacts}
\label{sec-3-7}
\begin{enumerate}
\item \textbf{target:} Save the same image in TIFF and JPEG formats. Compare
the differences
\item file > open : [tulip.tif] (IJ:26.2)
\item image > duplicate (IJ:28.9)
\begin{itemize}
\item work on the copy
\end{itemize}
\item file > save as > jpg (IJ:26.10.3)
\begin{itemize}
\item tulip.jpg
\end{itemize}
\item\relax [palete > pencil ] (IJ:19.19)
\begin{itemize}
\item modify value of only one pixel
\end{itemize}
\item file > save as > jpg (IJ:26.10.3)
\begin{itemize}
\item tulip-1px.jpg
\end{itemize}
\item close all jpeg files and reopen them
\item process > calculator (IJ:29.13)
\begin{itemize}
\item image1: tulip.tif
\item operation: subtract
\item image2: tulip.jpg
\item checked: create new window
\item checked: 32 bit result
\end{itemize}
\item process > calculator (IJ:29.13)
\begin{itemize}
\item image1: tulip-1px.jpg
\item operation: subtract
\item image2: tulip.jpg
\item checked: create new window
\item checked: 32 bit result
\end{itemize}
\item \textbf{questions:}
\begin{itemize}
\item What are the differences between images tif and jpg images? Why?
\item How many pixels are affected by changing only 1 pixel in jpg
image? Why?
\end{itemize}
\end{enumerate}
\section{background elimination}
\label{sec-4}
\subsection{dividing by background image}
\label{sec-4-1}
\begin{enumerate}
\item \textbf{target:} Estimate the local ratio of increase by dividing the
image by background. Check the impact of 32-bit image conversion
on the quatlity of the result.
\item file > open : [xxx.tif] (IJ:26.2)
\item file > open : [xxx$_{\text{background}}$.tif] (IJ:26.2)
\item process > calculator (IJ:29.13)
\begin{itemize}
\item image1: xxx.tif
\item operation: subtract
\item image2: xxx$_{\text{background}}$.tif
\item checked: create new window
\item checked: 32 bit result
\end{itemize}
\item process > calculator (IJ:29.13)
\begin{itemize}
\item image1: xxx.tif
\item operation: subtract
\item image2: xxx$_{\text{background}}$.tif
\item checked: create new window
\item unchecked: 32 bit result
\end{itemize}
\item \textbf{question:} what is the reason of \href{https://en.wikipedia.org/wiki/Posterization}{posterization} ?
\end{enumerate}

\subsection{background elimination - flat field correction}
\label{sec-4-2}
\begin{enumerate}
\item file > open > cell colony (IJ:26.4)
\item use selection to draw a horizontal line across the image (IJ:19.2)
\item analyze > plot profile
\begin{itemize}
\item check live
\end{itemize}
\item process > subtract background
\begin{itemize}
\item click: preview
\item click: create background
\item vary: rolling ball radius
\item try: sliding paraboloid
\end{itemize}
\end{enumerate}

\subsection{background elimination - flat field correction using Image calculator}
\label{sec-4-3}
\begin{enumerate}
\item file > open samples > cell colony (IJ:26.4)
\item image > duplicate (IJ:28.9)
\item process > filters > gaussian blur \% sigma \textasciitilde{}= 30 (IJ:29.11.2)
\item measure mean of blurred image (select it, "a", "m") (IJ:29.12.1)
\item process > calculator plus > divide (i1 = image, i2 = blurred image, k1 = mean, k2 = 0)
\end{enumerate}

\section{manual measurements and working with rois}
\label{sec-5}
\subsection{measuring fluorescence within a selection}
\label{sec-5-1}
\begin{enumerate}
\item task : measure average flu. in Red channel in neuron
\item file > open samples > neuron (IJ:26.4)
\item image > color > channel tool (IJ:28.7.5)
\begin{itemize}
\item split channels (IJ:28.5.1)
\end{itemize}
\item close all but green and red
\item work on green image
\begin{itemize}
\item image > duplicate (IJ:28.9)
\item process > filters > gaussian blur (IJ:29.11.2)
\begin{itemize}
\item use preview to set parameters
\end{itemize}
\item image > adjust > threshold (IJ:28.2.4)
\begin{itemize}
\item threshold to create neuron mask (avoid false negatives)
\end{itemize}
\item use wand tool to select main part of the neuron (IJ:19.7)
\item analyze > tools > roi manager (IJ:30.14.5)
\begin{itemize}
\item roi manager > add (t) (IJ:27.12.22)
\end{itemize}
\end{itemize}
\item choose second copy of green
\begin{itemize}
\item image > adjust > threshold (IJ:28.2.4)
\begin{itemize}
\item threshold to create neuron mask (avoid false negatives)
\end{itemize}
\item edit > selection > create selection (IJ:27.12.11)
\item roi manager > add (t) (IJ:27.12.22)
\item edit > selection > select none (ctrl-shift-a) (IJ:27.12.2)
\item process > noise > remove outliers (IJ:29.6.5)
\begin{itemize}
\item use preview; remove some of the outliers outside of neuron
\end{itemize}
\item edit > selection > create selection (IJ:27.12.11)
\item roi manager > add (t) (IJ:27.12.22)
\end{itemize}
\item analyze > set measurements (IJ:30.7)
\begin{itemize}
\item check: area / area fraction / mean gray value
\end{itemize}
\item work on red image
\begin{itemize}
\item choose multi point tool (IJ:19.5)
\begin{itemize}
\item select some points in the neuron
\item edit > selection > enlarge (IJ:27.12.14)
\item roi manager > add (t) (IJ:27.12.22)
\end{itemize}
\end{itemize}
\item for each selection
\begin{itemize}
\item analyze > measure (m) (IJ:29.12.1)
\end{itemize}
\item roi manager > more > save selection (IJ:30.14.5)
\end{enumerate}

\subsection{measuring geometrical properties in the image}
\label{sec-5-2}
\begin{enumerate}
\item task : measure average flu. in Red channel in neuron
\item file > open samples > neuron (IJ:26.4)
\item use polygon selection tool to measure cell body area (IJ:19.1.6)
\begin{itemize}
\item use measure to get the read out after creating polygon (IJ:29.12.1)
\begin{itemize}
\item roi manager > add (t) (IJ:27.12.22)
\end{itemize}
\end{itemize}
\item use segmented line tool (IJ:19.2.2) to measure length of few
dendrits
\begin{itemize}
\item test shift and alt while adding points (with mouse over a point)
\item use measure to get the read out after creating a line
(IJ:29.12.1)
\item roi manager > add (t) (IJ:27.12.22)
\end{itemize}
\item use angle tool (IJ:19.2.2) to measure length of few dendrits
\begin{itemize}
\item use measure to get the read out after creating an angle (IJ:29.12.1)
\item roi manager > add (t) (IJ:27.12.22)
\end{itemize}
\item roi manager > more > save selection (IJ:30.14.5)
\end{enumerate}

\section{automatic measurements}
\label{sec-6}
\subsection{identifying and measuring objects - basics}
\label{sec-6-1}
\begin{enumerate}
\item open : [blobs-thr.tif] (IJ:26.2)
\item image > duplicate (IJ:28.9)
\item process > binary > watershed (IJ:29.8.12)
\item process > noise > remove outliers (IJ:29.6.5)
\begin{itemize}
\item check preview
\end{itemize}
\item analyze > set measurements (IJ:30.7)
\begin{itemize}
\item check: area / area fraction / mean gray value
\end{itemize}
\item analyze > analyze particles  (IJ:30.2)
\begin{itemize}
\item test different options
\end{itemize}
\end{enumerate}

\subsection{identifying and measuring objects - cells \#1}
\label{sec-6-2}
\begin{enumerate}
\item \textbf{target:} measure distribution of RFP signal inside nucleus across cell population
\item open : [hela1.tif] (IJ:26.2)
\item image > adjust > threshold (IJ:28.2.4)
\begin{itemize}
\item test different option to isolate cells
\end{itemize}
\item \textbf{question:} What are the difficulties?
\end{enumerate}

\subsection{identifying and measuring objects - cells \#2}
\label{sec-6-3}
\begin{enumerate}
\item \textbf{target:} measure distribution of RFP signal inside nucleus
across cell population
\item open : [hela2.tif] (IJ:26.2)
\item image > color > split channel (IJ:28.5.1)
\item work on blue channel (DAPI)
\begin{itemize}
\item image > adjust > threshold (IJ:28.2.4)
\item process > binary > watershed (IJ:29.8.12)
\item process > noise > remove outliers (IJ:29.6.5)
\begin{itemize}
\item check preview
\end{itemize}
\end{itemize}
\item analyze > set measurements (IJ:30.7)
\begin{itemize}
\item check: area / area fraction / mean gray value
\item redirect to: RFP
\end{itemize}
\item analyze > analyze particles  (IJ:30.2)
\item analyze > distribution (IJ:30.4)
\begin{itemize}
\item choose: gray value
\end{itemize}
\end{enumerate}

\section{scripting / macros / automation basics}
\label{sec-7}
\subsection{macro recorder - reproducing edits}
\label{sec-7-1}
\begin{enumerate}
\item \textbf{target:} save the edit chain for later (to save work and to
document parameters used by different filters)
\item plugins > macro > recorder (IJ:31.1.4)
\item file > open samples > mri-stack (IJ:26.4)
\item process > filters > gaussian blur \% sigma \textasciitilde{}= 2 (IJ:29.11.2)
\item image > adjust > threshold (IJ:28.2.4)
\begin{itemize}
\item select manually threshold such as the head is separated from background
\item ignore small holes
\item uncheck box: calculate threshold for each image
\end{itemize}
\item process > binary > fill holes (IJ:29.8.8)
\item (optional) use analyze > set scale to calibrate the measurement units
\begin{itemize}
\item 200 pixels is 25 cm
\end{itemize}
\end{enumerate}
\subsection{macro recorder - repeating actions on stack}
\label{sec-7-2}
\begin{enumerate}
\item \textbf{target:} calculate the volume of human skull
\item file > open samples > mri-stack (IJ:26.4)
\begin{itemize}
\item use steps below or macro developed in previous excercise
\end{itemize}
\item process > filters > gaussian blur \% sigma \textasciitilde{}= 2 (IJ:29.11.2)
\item image > adjust > threshold (IJ:28.2.4)
\begin{itemize}
\item select manually threshold such as the head is separated from background
\item ignore small holes
\end{itemize}
\item process > binary > fill holes (IJ:29.8.8)
\item (optional) use set scale to calibrate the measurement units
\item plugins > macro > recorder (IJ:31.1.4)
\item analyse > measure (IJ:29.12.1)
\item image > stacks > next slice (IJ:28.6.3)
\item in recorder click "create" button
\item use copy paste to execute the same action many times:
\end{enumerate}
\begin{verbatim}
run("Next Slice [>]");
run("Measure");
run("Next Slice [>]");
run("Measure");
run("Next Slice [>]");
run("Measure");
run("Next Slice [>]");
run("Measure");
\end{verbatim}
\begin{enumerate}
\item \textbf{Question:} how to compute volume of human skull based on the measurement?
\end{enumerate}

\subsection{automatic iteration}
\label{sec-7-3}
\begin{enumerate}
\item \textbf{target:} avoid copy pasting - use iteration instead
\item continue from last example or open preprocessed file
\begin{itemize}
\item file > open > [mri-stack-binary] (IJ:26.4)
\end{itemize}
\item make sure that the code looks in the following way:
\end{enumerate}
\begin{verbatim}
for (currentStep=0; currentStep<100;currentStep++){
  run("Next Slice [>]");
  run("Measure");
}
\end{verbatim}
\begin{enumerate}
\item \textbf{Question:} what is the issue with this approach?
\end{enumerate}

\subsection{automatic iteration - introducing function for stop condition}
\label{sec-7-4}
\begin{enumerate}
\item \textbf{target:} avoid fixed stop condition and exchange it by more
accurate mechanism
\item continue from last example or open preprocessed file
\begin{itemize}
\item file > open > [mri-stack-binary] (IJ:26.4)
\end{itemize}
\item make sure that the code looks in the following way:
\end{enumerate}
\begin{verbatim}
for (currentStep=1; currentStep<=nSlices();currentStep++){
  setSlice(currentStep);
  run("Measure");
}
\end{verbatim}

\subsection{adding macro to the menu}
\label{sec-7-5}
\begin{enumerate}
\item \textbf{target:} add macro to the menu
\item continue from last example or open preprocessed file
\begin{itemize}
\item file > open > [mri-stack-binary] (IJ:26.4)
\end{itemize}
\item switch to code editor and make sure that the code looks in the
following way:
\end{enumerate}
\begin{verbatim}
macro "measure stack" {
  for (currentStep=1; currentStep<=nSlices();currentStep++){
	setSlice(currentStep);
	run("Measure");
  }
}
\end{verbatim}
\begin{enumerate}
\item language > ImageJ Macro
\item save > .ijm
\item plugins > macros > install\ldots{} (IJ:31.1.1) 
\begin{itemize}
\item select the macro you just saved
\end{itemize}
\item (optional) test following FIJI tool
\begin{itemize}
\item image > stacks > plot z axis profile (IJ:28.6.13)
\end{itemize}
\end{enumerate}

\subsection{iterations and variables in the macro: multi-measurement}
\label{sec-7-6}
\begin{enumerate}
\item \textbf{target:} prepare the evenly distributed selection over the
image v.1 This can be used e.g. in FRAP experiment analysis.
\item open : [frap.tif] (IJ:26.2)
\item test code:
\end{enumerate}
\begin{verbatim}
// initialization
x = 20;
y = 20;

// iterations
for (nbr_x=0;nbr_x<5;nbr_x++){
  for (nbr_y=0;nbr_y<5;nbr_y++){
	makePoint(x+x*nbr_x, y+y*nbr_y);
	run("Enlarge...", "enlarge=5 pixel");
	roiManager("Add");
  }
}
\end{verbatim}
\subsection{automatic measurement and saving to a file}
\label{sec-7-7}
\begin{enumerate}
\item \textbf{target:} prepare the evenly distributed selection over the image v.2
\item test code:
\end{enumerate}
\begin{verbatim}
// initialization
roiManager("reset");
x = 220;
y = 110;
nbr_sensors = 4;
delta = 12;

// iterations to create selection
for (nbr_x=0;nbr_x<nbr_sensors;nbr_x++){
  for (nbr_y=0;nbr_y<nbr_sensors;nbr_y++){
	makePoint(x+delta*nbr_x, y+delta*nbr_y);
	run("Enlarge...", "enlarge=5 pixel");
	roiManager("Add");
  }
}

// measurement and save results
roiManager("Multi Measure");
fn = getInfo("image.filename");
saveAs("Results", "/Users/sstoma/Desktop/Results-"+ fn +".txt");
// exchange previous line with your path
\end{verbatim}

\subsection{user input via GUI}
\label{sec-7-8}
\begin{enumerate}
\item \textbf{target:} prepare the evenly distributed selection over the image v.3
\item test code:
\end{enumerate}
\begin{verbatim}
// initialization
roiManager("reset");
x = 20;
y = 20;
nbr_sensors = 4;
delta = 12;

// GUI
Dialog.create("Please specify parameters:");
Dialog.addNumber("Size [px]: ", 5);
Dialog.show();
size = Dialog.getNumber();

// iterations to create selection
for (nbr_x=0;nbr_x<nbr_sensors;nbr_x++){
  for (nbr_y=0;nbr_y<nbr_sensors;nbr_y++){
	makePoint(x+delta*nbr_x, y+delta*nbr_y);
	run("Enlarge...", "enlarge="+ size +" pixel");
	roiManager("Add");
  }
}

// measurement and save results
roiManager("Multi Measure");
fn = getInfo("image.filename");
saveAs("Results", "/Users/sstoma/Desktop/Results-"+ fn +".txt");
// exchange previous line with your path
\end{verbatim}

\subsection{batch mode - basics}
\label{sec-7-9}
\begin{enumerate}
\item \textbf{target:} Create a macro to segment nuclei in a single
frame. Your macro should input a row to the Results containing
area for each nuclei in the image. Process all files in the hela
folder.
\item open : [hela/h01.tif] (IJ:26.2)
\item plugins > macro > recorder (IJ:31.1.4)
\item image > adjust > threshold (IJ:28.2.4)
\begin{itemize}
\item select: triangle
\item check: dark background
\end{itemize}
\item analyze > analyze particles  (IJ:30.2)
\item make sure that the code looks in the following way:
\end{enumerate}
\begin{verbatim}
// initialization
outPath = "/Users/sstoma/Desktop/materials/images/hela/out/"; 
// exchange previous line with your path

// processing
run("Gaussian Blur...", "sigma=2 stack");
setAutoThreshold("Triangle dark");
run("Convert to Mask");
run("Set Measurements...", "area mean display redirect=None decimal=9");
run("Analyze Particles...", "show=[Overlay Masks] display exclude");

// saving
saveAs("Results", outPath+"Results.txt");
\end{verbatim}
\begin{enumerate}
\item process > batch > macro
\begin{itemize}
\item input: select folder with [h01.tif]
\item output: select folder out in the folder containing [h01.tif]
\end{itemize}
\item click process

\item modify the code (\textbf{target:} save each result in separated file):
\end{enumerate}
\begin{verbatim}
// initilization
fileName = getInfo("image.filename");
outPath = "/Users/sstoma/Desktop/materials/images/hela/out/"; 
// exchange previous line with your path
run("Clear Results");

// processing
run("Gaussian Blur...", "sigma=2 stack");
setAutoThreshold("Triangle dark");
run("Convert to Mask");
run("Set Measurements...", "area mean display redirect=None decimal=9");
run("Analyze Particles...", "show=[Overlay Masks] display exclude");

// saving
saveAs("Results", outPath + "Results-" + fileName + ".txt");
\end{verbatim}
\section{workflow: tracking}
\label{sec-8}
\subsection{create image with moving dots}
\label{sec-8-1}
\begin{enumerate}
\item \textbf{target:} create an image with moving objects (dots)
\item make sure that the code looks in the following way:
\end{enumerate}
\begin{verbatim}
macro "create_image_with_moving_objects"{
  // initial variables
  nbr_frames = 40; // number of frames in the image
  color1 = 150; // color of first obj.
  color2 = 250; // color of sec. obj.
  canvas_size = 200; // size of the image
  step = 4; // progress in x between frames
  delta = 15; // difference in position between two objects in x
  x = 5; // initial position x of object
  y = 5;  // initial position y of object
  width = 10; // baseline size of object in x
  height = 10; // baseline size of object in y

  // empty image with noise
  newImage("Image", "8-bit black", canvas_size, canvas_size, nbr_frames); // image will be named "Image"
  run("Salt and Pepper", "stack");

  for(i=1;i<=nbr_frames;i++){
	setSlice(i);
	// first dot
	setColor(color1,color1,color1);
	fillOval(x+i*step, y+i*step, width, height);
	// second dot
	setColor(color2,color2,color2);
	fillOval(delta+x+i*step, canvas_size-(y+i*step), width*2, height*2);
  }
  run("glasbey");
}
\end{verbatim}

\subsection{tracking: process single image}
\label{sec-8-2}
\begin{enumerate}
\item \textbf{target:} for \textbf{current} frame of the stack find the x, y of the
center of the dot
\item make sure that the code looks in the following way
\end{enumerate}
\begin{verbatim}
macro "tracking_process_single_image"{
  run("Median...", "radius=2 slice");
  setThreshold(0, 100);
  run("Convert to Mask", "method=Otsu background=Light only");
  run("glasbey");
  run("Find Maxima...", "noise=25 output=List light");
}
\end{verbatim}

\subsection{tracking: process whole stack}
\label{sec-8-3}
\begin{enumerate}
\item \textbf{target:} for \textbf{all} frames of the stack find the x, y of the
center of the dot
\item make sure that the code looks in the following way:
\end{enumerate}
\begin{verbatim}
macro "tracking_process_stack_v1"{
  r = newArray();
  for (i=1;i<=nSlices();i++){
	setSlice(i);
	run("Median...", "radius=2 slice");
	setThreshold(0, 100);
	run("Convert to Mask", "method=Otsu background=Light only");
	run("glasbey");
	run("Find Maxima...", "noise=25 output=List light");
	run("Next Slice [>]");

  // workaround for Find maxima overwriting results at each step
	for (j=0;j<nResults();j++){
	  x = getResult("X", j);
	  y = getResult("Y", j);
	  print(x, y, i);
	  temp = newArray(x, y);
	  r = Array.concat(r, temp);
	}
  }
  Array.show( r );
}
\end{verbatim}

\subsection{tracking: process whole stack - enabling linking of the objects}
\label{sec-8-4}
\begin{enumerate}
\item \textbf{target:} for \textbf{all} frames of the stack find the x, y of the
center of the dot as well as some object features (i.e. size,
mean grey value).
\item make sure that the code looks in the following way:
\end{enumerate}
\begin{verbatim}
macro "tracking_process_stack_v2"{
  // clearing previous result and preparing image copy
  run("Clear Results"); 
  rename("Image");
  run("Duplicate...", "duplicate");
  rename("orig");
  selectWindow("Image");

  // iterating for each slice in the stack
  for (i=1;i<=nSlices();i++){
	setSlice(i);
	run("Median...", "radius=2 slice"); // removing noise
	setThreshold(0, 100); // hardcoded thr. for image
	run("Convert to Mask", "method=Otsu background=Light only");
	run("glasbey"); // changing LUT to get false collors to better distinguish objects
	run("Set Measurements...", "area mean standard center median skewness area_fraction stack redirect=orig decimal=2");
	run("Analyze Particles...", "display slice");
	run("Next Slice [>]");
  }
}
\end{verbatim}
\section{workflow: FRET}
\label{sec-9}
\subsection{preparation part 1 - image import}
\label{sec-9-1}
\begin{enumerate}
\item \textbf{target:} open .lif image and prepare it for further editing
\item file > open : [FRET$_{\text{biosensor}}$.lif] (IJ:26.2)
\begin{itemize}
\item configure the importer
\item disable: all series
\end{itemize}
\item choose 3rd serie
\item image > stacks > z project (IJ:28.6.11)
\item image > color > split channel (IJ:28.5.1)
\item select blue channel (ch0) 
\begin{enumerate}
\item file > save as > tif : cfp.tif (IJ:26.10.1)
\end{enumerate}
\item select yellow channel (ch1) 
\begin{enumerate}
\item file > save as > tif : yfp.tif (IJ:26.10.1)
\end{enumerate}
\end{enumerate}

\subsection{preparation part 2 - averaging}
\label{sec-9-2}
\begin{enumerate}
\item \textbf{target:} prepare images for further editing
\item file > open : [cfp.tif and yfp.tif] (IJ:26.2)
\item select blue channel (cfp.tif)
\begin{enumerate}
\item process > filters > gaussian blur \% sigma \textasciitilde{}= 2 (IJ:29.11.2)
\item file > save as > tif : cfp-smoothed.tif (IJ:26.10.1)
\end{enumerate}
\item select yellow channel (yfp.tif)
\begin{enumerate}
\item process > filters > gaussian blur \% sigma \textasciitilde{}= 2 (IJ:29.11.2)
\item file > save as > tif : yfp-smoothed.tif (IJ:26.10.1)
\end{enumerate}
\end{enumerate}

\subsection{preparation part 3 - masks}
\label{sec-9-3}
\begin{enumerate}
\item \textbf{target:} prepare masks
\item file > open : [yfp-smoothed.tif] (IJ:26.2)
\item image > adjust > threshold (IJ:28.2.4)
\begin{itemize}
\item test different option to isolate chromatine
\end{itemize}
\item file > save as > tif : mask.tif (IJ:26.10.1)
\end{enumerate}

\subsection{preparation part 4 - ratios}
\label{sec-9-4}
\begin{enumerate}
\item \textbf{target:} prepare image with ratios
\item file > open : [cfp-smoothed.tif and yfp-smoothed.tif] (IJ:26.2)
\item process > calculator plus > divide (i1 = yfp, i2 = cfp, k1 = 1, k2
= 0)
\item file > save as > tif : ratio.tif (IJ:26.10.1)
\end{enumerate}

\subsection{analysis - problem: cell population}
\label{sec-9-5}
\begin{enumerate}
\item \textbf{target:} measure the change of signal in ratio.tif image
\item file > open : [ratio.tif] (IJ:26.2)
\item analyze > set measurements (IJ:30.7)
\begin{itemize}
\item check: mean gray value / standard deviation
\end{itemize}
\item analyze > measure (m) (IJ:29.12.1)
\item move to next time point; repeat;\ldots{}
\item image > stacks > plot z axis profile (IJ:28.6.13)
\begin{itemize}
\item if does not work: image > hyperstacks > re-order hyperstack
\begin{itemize}
\item swap t with z
\end{itemize}
\end{itemize}
\item question:  what is the problem with results?
\end{enumerate}

\subsection{preparation part 5 - cropping}
\label{sec-9-6}
\begin{enumerate}
\item \textbf{target:} prepare image with ratios
\item file > open : [all previously prepared images] (IJ:26.2)
\item choose a cell in transition to anaphase - make sure the field of
view keeps only this cell during all time-points
\begin{itemize}
\item\relax [palette > rectangular selection] (IJ:19.1)
\item edit > selection > add to manager (ctrl-t) (IJ:27.12.22)
\item use roi manager to move selection between images
\begin{itemize}
\item analyze > tools > ROI manager
\end{itemize}
\end{itemize}
\item image > crop (IJ:28.8)
\item save all files adding "-crop" postfix
\end{enumerate}

\subsection{analysis - problem: chromatin change}
\label{sec-9-7}
\begin{enumerate}
\item \textbf{target:} measure the change of signal in ratio-crop.tif image
\item file > open : [ratio-crop.tif] (IJ:26.2)
\item analyze > set measurements (IJ:30.7)
\begin{itemize}
\item check: mean gray value / standard deviation
\end{itemize}
\item analyze > measure (m) (IJ:29.12.1)
\item move to next time point; repeat;\ldots{}
\item image > stacks > plot z axis profile (IJ:28.6.13)
\begin{itemize}
\item if does not work: image > hyperstacks > re-order hyperstack
\begin{itemize}
\item swap t with z
\end{itemize}
\end{itemize}
\item question: what are the issues with results?
\end{enumerate}

\subsection{analysis - problem: manual labor}
\label{sec-9-8}
\begin{enumerate}
\item \textbf{target:} measure the change of signal in ratio-crop.tif image limited
to chromatin
\item file > open : [mask-crop.tif] (IJ:26.2)
\item\relax [palette > wand selection] (IJ:19.7)
\begin{itemize}
\item add chromatin from current time-point
\item edit > selection > add to manager (ctrl-t) (IJ:27.12.22)
\item move to next time-point
\item repeat
\end{itemize}
\item file > open : [ratio-crop.tif] (IJ:26.2)
\item analyze > set measurements (IJ:30.7)
\begin{itemize}
\item check: mean gray value / standard deviation
\end{itemize}
\item select the right ROI; analyze > measure (m) (IJ:29.12.1)
\item move to next time point; repeat;\ldots{}
\item question: what are the issues with results?
\end{enumerate}
\section{various useful tools}
\label{sec-10}
\subsection{installing plugins}
\label{sec-10-1}
\begin{enumerate}
\item download plugin from webpage:
\url{http://bigwww.epfl.ch/algorithms/esnake/}
\item unzip, drag and drop to FIJI
\item create new canvas
\item draw two white discs on black background
\item plugins > ESnake
\begin{itemize}
\item target brightness: bright
\end{itemize}
\item click OK
\end{enumerate}

\subsection{using line selection to make a "straighten" image}
\label{sec-10-2}
\begin{enumerate}
\item file > open samples > nile bend (IJ:26.4)
\item use selection tool for freehand selection
\begin{itemize}
\item make the line thickness adjusted to cover whole river
\end{itemize}
\item edit > selection > straighten (IJ:27.12.17)
\end{enumerate}

\subsection{using 3D viewer}
\label{sec-10-3}
\begin{enumerate}
\item file > open samples > confocal series (IJ:26.4)
\item image > properties > voxel depth x10 \% to get decent aspect ratio
\item plugins > 3D viewer
\item add > from image \% the resampling factor is a downsampling factor
\item play with displayed colors
\item view > start/stop animation
\item view > change animation settings
\item view > record 360 degree rotation
\item file > save as > avi \% try the different compression options uncompressed, jpg, and png
\end{enumerate}
% Emacs 25.0.50.4 (Org mode 8.2.10)
\end{document}